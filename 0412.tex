%%%%LEZIONE 12 APRILE%%%%
Lezione del 12/04, ultima modifica 20/05, Andrea Gadotti\\
\\
\\

\textbf{Esempio di test t-Student per due campioni}\\
Campione $(X_1,...,X_{n_1})$ da $N(\mu_1,\sigma^2)$ e $(Y_1,...,Y_{n_2})$ da $N(\mu_2,\sigma^2)$ indipendenti. La varianza $\sigma$ è la stessa ma non è nota.\\
Supponiamo di avere elementi per pensare che:
$$\bigg \{
\begin{array}{rl}
H_0: & \mu_1=\mu_2 \\
H_1: & \mu_1>\mu_2 \\
\end{array}
$$
Abbiamo che $\overline{X}_1 - \overline{X}_2 \sim N(\mu_1 - \mu_2, \sigma_1/n_1 + \sigma_2/n_2)$.\\
Prendiamo ora 
$$S_p^2:= \frac{(n_1-1)S_1^2 + (n_2-1)S_2^2}{n-2}$$ 
dove $n=n_1+n_2$.\\
Abbiamo che:
$$T:= \frac{(\overline{X}_1 - \overline{X}_2)-(\mu_1-\mu_2)}{S_p \sqrt{1/n_1+1/n_2}} \; \stackrel{H_0}{\sim} \; t_{n-2}$$
dove abbiamo usato anche il fatto che $S_p^2$ è una chi-quadro con $n-2$ gradi di libertà (vedi pag. 19).\\
In conclusione, avremo che la nostra regione critica sarà $C=\lbrace (\underline{x},\underline{y}) \mid T \geq t_{n-2;\alpha} \rbrace$.\\
\\
\textbf{Esempio con bernoulliana} Abbiamo $(X_1,...,X_n)$ campione casuale da $b(1,p)$ (ricordiamo: media $p$, varianza $p(1-p)$). Le nostre ipotesi sono:
$$\bigg \{
\begin{array}{rl}
H_0: & p=p_0 \\
H_1: & p=p_1 \\
\end{array}
$$
con $p_1<p_0$. Consideriamo
$$\hat{p}_n := \frac{\sum X_i}{n} \stackrel{a}{\sim} N \left( p, \frac{p(1-p)}{n} \right)$$
Abbiamo quindi:
$$Z:= \frac{\hat{p}_n-p_0}{\sqrt{\frac{\hat{p}_n (1-\hat{p}_n)}{n}}}$$
A questo punto imponiamo: $\alpha = P(Z \leq z_{\alpha} \mid H_0)$.\\
\\
\textbf{Esempio di test sulla varianza} Campione casuale $(X_1,...,X_n)$ da $N(\mu,\sigma^2)$, $\mu$ e $\sigma^2$ non noti. Le nostre ipotesi sono: 
$$\bigg \{
\begin{array}{rl}
H_0: & \sigma^2=\sigma_0^2 \\
H_1: & \sigma^2=\sigma_1^2 \\
\end{array}
$$
con $\sigma_1^2 > \sigma_0^2$.
Notiamo che $\frac{n-1}{\sigma_0^2} S_n^2 \stackrel{H_0}{\sim} \chi_{n-1}^2 =: W$. (Nota: come in molti esempi precedenti, il fatto che sia "intelligente" tirare fuori queste osservazioni che portano all'analisi di distribuzioni conosciute è "calato dall'alto", almeno per ora).\\
Imponiamo:
$$\alpha = P \left( \frac{n-1}{\sigma^2} S_n^2 \mid \sigma^2=\sigma_0^2 \right) = P \left( \frac{n-1}{\sigma_0^2} S_n^2 \right)$$
Quindi $k=w_{\alpha;n-1}$. In conclusione, rifiuto $H_0$ se $W \geq w_{\alpha;n-1}$.\\
\textbf{Esempio} In riferimento al caso generale appena trattato, supponiamo di avere $n=25$, $\sigma_0^2=15$, $\sigma_1^2=20$, $s_n^2=17,4$ e $\alpha=0,05$. Allora:
$$k=w_{0,05;(25-1)} = w_{0,05;24} = 36,415 > 27,84 = \frac{25-1}{15} 17,4 = \frac{n-1}{\sigma_0^2} s_n^2 = w$$
In conclusione, non rifiutiamo $H_0$.\\
\\
\textbf{Esempio con due campioni normali} Campione $(X_1,...,X_{n_1})$ da $N(\mu_1,\sigma_1^2)$ e $(Y_1,...,Y_{n_2})$ da $N(\mu_2,\sigma_2^2)$ indipendenti. $\mu_i$ e $\sigma_i^2$ non noti. Le nostre ipotesi sono:
$$\bigg \{
\begin{array}{rl}
H_0: & \sigma_1^2=\sigma_2^2 \\
H_1: & \sigma_1^2 \neq \sigma_2^2 \\
\end{array}
$$
Nota: l'ipotesi dell'uguaglianza delle due varianze prende il nome di omoschedasticità.\\
Sappiamo che $S_i^2 \stackrel{P}{\rightarrow} \sigma_i^2$. Inoltre
$$W:= \frac{S_2^2 / \sigma_2^2}{S_1^2 / \sigma_1^2} = 
\frac{\frac{(n_2-1) S_2^2}{\sigma_2^2} \frac{1}{n_2-1}} {\frac{(n_1-1) S_1^2}{\sigma_1^2} \frac{1}{n_1-1}} 
\sim F_{(n_2-1),(n_1-1)}$$
(vedi pag. 20)\\
Sotto $H_0$ abbiamo chiaramente che $W:= \frac{S_2^2 / \sigma_2^2}{S_1^2 / \sigma_1^2} = \frac{S_2^2}{S_1^2}$. La nostra regola di decisione consisterà nel rifiutare $H_0$ a favore di $H_1$ se $\frac{S_2^2}{S_1^2}$ è ''lontano'' da 1, ovvero se $W<k_1$ o $W>k_2$, dove $k_1$ e $k_2$ dipendono dalla distribuzione di $W= \frac{S_2^2}{S_1^2}$ e dal valore di $\alpha$. Perciò, dividendo equamente in due parti la probabilità di errore, le due equazioni risultano:
$$\alpha/2 = P(W>k_2 \mid \sigma_1^2=\sigma_2^2) \; \; \text{ e } \; \; 1-\alpha/2 = P(W<k_1 \mid \sigma_1^2=\sigma_2^2)$$
In conclusione, 
$$C= \lbrace (\underline{x_1},\underline{x_2}) \; : \; \frac{S_2^2}{S_1^2} < w_{(n_2-1),(n_1-1);\alpha/2} \; \; \text{ o } \; \; \frac{S_2^2}{S_1^2} > w_{(n_2-1),(n_1-1);1-\alpha/2}$$
In riferimento al caso generale appena trattato, supponiamo di avere $n_1=14$, $n_2=10$ $s_1^2=17,4$, $\sigma_1^2=20$, $s_2^2=37,9$ e $\alpha=0,05$.\\
Come nel caso generale, diciamo $W:=S_2^2/S_1^2 \sim F_{(n_2-1),(n_1-1)}$.\\
Abbiamo che $w_{(10-1),(14-1);0,025}=3,31$ e $w_{(10-1),(14-1);0,975}=1/w_{13,19;0,025}=1/3,76=0,26$.\\
Poiché $s_2^2/s_1^2=37,9/17,4=2,178$, decidiamo di non rifiutare $H_0$.