%%%%lezione 13 maggio%%%%

Lezione del 13/05, ultima modifica 16/06, Andrea Gadotti

\subsection{Famiglie esponenziali e sufficienza}

Notiamo innanzitutto che una qualsiasi funzione di densità per una distribuzione appartenente alla famiglia esponenziale può essere scritta come
$$f_X(x;\theta) = \exp \left\lbrace C(x) + D(\theta) \displaystyle \sum_{m=1}^k A_m(\theta) B_m(x) \right\rbrace$$

\begin{teo}

Sia $(X_1,...,X_n)$ un campione casuale appartenente alla famiglia esponenziale a k parametri. Allora il vettore $T_n:=(\sum_{i=1}^n B_1(x_i),...,\sum_{i=1}^n B_k(x_i))$ è statistica sufficiente per $\theta$.

\begin{proof}
Da inserire (comunque è facile, basta usare il teorema di fattorizzazione).
\end{proof}

\end{teo}

\begin{esempio}
Sia $(X_1,...X_n)$ da $N(\mu, \sigma^2)$ con $\mu$ e $\sigma^2$ non noti (dunque $\vec{X}$ proviene da una famiglia esponenziale a 2 parametri). Per il teorema appena visto si ha che $I_n:= (\sum_{i=1}^n B_1(x_i), \sum_{i=1}^n B_2(x_i)) = (\sum_{i=1}^n x_i, \sum_{i=1}^n x_i^2$ è statistica sufficiente per $(\mu, \sigma^2)$. (Nota: la seconda uguaglianza si verifica immediatamente).\\
Affermiamo (senza dimostrazione) che lo stimatore di massima verosimiglianza per $(\mu, \sigma^2)$ è $\hat{\theta} = \left( \frac{1}{n} \sum_{i=1}^n X_i, \frac{1}{n} \sum_{i=1}^n (X_i - \overline{X}_n)^2 \right)$ e notiamo che esso è funzione della statistica sufficiente $I_n$.\\
\\
\end{esempio}